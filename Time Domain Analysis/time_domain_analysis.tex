\documentclass{article}

\usepackage{fullpage}
\usepackage{tikz}
\usepackage[siunitx, RPvoltages]{circuitikz}
\usepackage{amsmath}
\usepackage{hyperref}
\usepackage{chngcntr}

\counterwithin{figure}{section}

\hypersetup{
	colorlinks,
	citecolor=black,
	filecolor=black,
	linkcolor=blue,
	urlcolor=cyan,
}

\title{\textbf{Time Domain Analysis}}
\author{Jay Khandkar}
\date{January 18, 2021}

\begin{document}

\maketitle
\newpage
\tableofcontents
\newpage

\color{blue}
\section{Current And Voltage Conventions}
\color{black}

The conventions we shall follow, simply state that when 
current "flows into" the positive terminal of the capacitor/inductor, as indicated by the polarity of $v$
in Fig.\ref{fig:conventions}, it is taken as positive. We may then write
\begin{align*}
i &= C\frac{dv}{dt}\\
v &= L\frac{di}{dt}
\end{align*}

\begin{figure}[h!]
\centering
\fbox{
\begin{circuitikz}[american]
	
	\draw (0,0) to[short, i=$i$] (1,0) to[C, l=$C$, v_=$v$, o-o, voltage shift=2.5] (2,0) 
	to[short] (3,0);
	\draw (0,-2) to[short, i=$i$] (1,-2) to[cute inductor, l=$L$, v^=$v$, o-o, voltage shift=5.0] (2,-2) 		to[short] (3,-2);
	
\end{circuitikz}
}
\caption{The voltage conventions}
\label{fig:conventions}	
\end{figure}

\color{blue}
\section{First Order Circuits: RL and RC}
\color{black}
A first order circuit is one that is governed by a first order differential equation. Often, it is
incorrectly stated that a first order circuit is one that contains only one energy storage element
(capacitor/inductor). This is wrong, as there are certain arrangements of $R-L$ and $R-C$ circuits
which can be simplified to obtain a first order equation, as we shall see later.

\color{blue}
\subsection{RL Circuits}
\subsubsection{The Natural Response}
\color{black}
A \textbf{\textit{natural response}} is one that is free of any external voltage/current sources, 
which are also known as \textit{forcing functions}. It depends on the "general nature" of the circuit (types of elements, sizes and interconnections). It is also known as the \textbf{\textit{transient response}}, as without any external sources, it must eventually die out. Consider the simple series RL
circuit shown in 

\end{document}
